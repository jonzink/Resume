\documentclass[margin,line, 12pt]{res}

\topmargin=-0.5in    % Make letterhead start about 1 inch from top of page
\textheight=10in  % text height can be bigger for a longer letter
\oddsidemargin -.5in
\evensidemargin -.5in
\textwidth=6.5in
\itemsep=0in
\parsep=0in

\nofiles

% if using pdflatex:
\setlength{\pdfpagewidth}{\paperwidth}
\setlength{\pdfpageheight}{\paperheight}

\newenvironment{list1}{
  \begin{list}{\ding{113}}{%
      \setlength{\itemsep}{0in}
      \setlength{\parsep}{0in} \setlength{\parskip}{0in}
      \setlength{\topsep}{0.05in} \setlength{\partopsep}{0in}
      \setlength{\leftmargin}{0.17in}}}{\end{list}}
\newenvironment{list2}{
  \begin{list}{$\bullet$}{%
      \setlength{\itemsep}{0.0in}
      \setlength{\parsep}{0in} \setlength{\parskip}{0in}
      \setlength{\topsep}{0.0in} \setlength{\partopsep}{0in}
      \setlength{\leftmargin}{0.2in}}}{\end{list}}

\renewcommand{\familydefault}{\sfdefault}
%\usepackage[sfdefault, light, condensed]{roboto}
\usepackage[light,condensed]{iwona}
\usepackage[T1]{fontenc}
\usepackage[colorlinks=false,urlcolor=magenta,citecolor=blue,linkcolor=blue]{hyperref}
\usepackage{setspace}






\begin{document}
\begin{spacing}{1.0}	

\name{{\huge\bf Jon Zink, Ph.D}\vspace*{.1in}}

\begin{resume}
\vspace*{-2mm}
\section{Basic\\Information}
\begin{tabular}{@{}p{4.75in}p{4in}}
  (248)909-1107 &  \href{https://www.jonzink.com}{jonzink.com}  \\
  \href{mailto:jonzink123@gmail.com}{jonzink123@gmail.com} & \href{https://github.com/jonzink}{github.com/jonzink} \\
  
  & \href{https://www.linkedin.com/in/jon-zink-phd/}{linkedin.com/in/jon-zink-phd} \\
\end{tabular}

% \vspace{-3mm}
% Collaborative, scientific thinker passionate about discovering and communicating nuanced insight from complicated data. Background includes: open-source contributions, project leadership, computer vision, traditional machine learning, and working with large, heterogeneous, often noisy datasets.
% \vspace*{-2mm}

\section{Professional \newline Experience}
% \textbf{Activision Publishing, Inc.} \hfill Boulder, Colorado\newline
% \textit{Senior Machine Learning Engineer} \hfill \textbf{January, 2021 - Present}\newline
% As part of the Advanced Analytics and Machine Learning (ML) team, I support all ML initiatives for the Call of Duty and Warzone franchises. This includes being involved in early concept designs through the productionalization of mature models and stakeholder management. Below are selected highlights from this very diverse role:
%     \begin{list2}
% %     	\vspace*{-5mm}
%       \item Designed, built, maintained, and improved ML infrastructure. This includes tool developement like autoML, CI/CD (Jenkins), model tracking and management (MLflow) and orchestration (Airflow).
%       \item Crafted and implimented policy around external data ingestion and data product retention for both my internal team and for sharing across the organization.
%       \item Designed, built, and maintained near-realtime applications (via Spark streaming) to combat in-game cheating and other malicious behavior. Typical time-to-action is in the low 10s of seconds.
%       \item Lead the transition of ML infrastructure from AWS to GCP/GCS.
%       \item Created ML models to provide insights into customer conversion, churn, and behavioral segmentation. This leveraged survival analysis, clustering, as well as tree-based and linear methods.
%       \item Supervised junior team members to design and develope recommendation systems to be productionalized in an upcoming title.
%     \end{list2}
% \vspace*{-2mm}
%
% \textbf{Insight Data Science} \hfill New York, New York\newline
% \textit{Fellow} \hfill \textbf{January, 2020 - 2021}\newline
%     \begin{list2}
%     	\vspace*{-5mm}
%       \item Helped optimize the way NYC health inspectors perform restaurant inspections in order to reduce the time critical health violations remain unaddressed.
%     	\item Trained a random forest in Python to prioritize NYC restaurant inspections based on environmental variables and their past inspection histories and provided the results to NYC through an API deployed on AWS.
%     	\item Resulted in NYC inspectors identifying $\sim$2.5\% more violations in the first half of an inspection window,  leading to critical violations being discovered up to 7 days earlier than by the current approach implemented by NYC.
%     \end{list2}
% \vspace*{-2mm}

\textbf{California Institute of Technology} \hfill \newline
\textit{NASA Hubble Postdoctoral Fellow} \hfill \textbf{September, 2021 - Present}\newline
    \begin{list2}
    	\vspace*{-5mm}
    	\item Extracted exoplanet population trend with 99.99\% confidence by accounting for sample biases/reliability, revealing the non-uniform distribution of planets within the Milky Way: see \href{https://exoplanets.nasa.gov/news/1768/discovery-alert-on-our-galaxys-outskirts-a-poverty-of-planets/}{NASA press release}.
	
	% . I showed that planets are not uniformly distributed about the Milky-way and was highlighted in a \href{https://exoplanets.nasa.gov/news/1768/discovery-alert-on-our-galaxys-outskirts-a-poverty-of-planets/}{NASA press release}.
	
	% \item Utilized statistical methods to extract population trends with 99.99\% confidence from an exoplanet dataset, revealing the non-uniform distribution of planets within the Milky Way galaxy, subsequently recognized by NASA and highlighted in a press release.
	\item Executed k-means clustering analysis to group planet classes and resolve a physically-driven partition, reducing the measurement uncertainty of existing methods by 22\%. 
	
	\item Resolved a 30-year-old planetary formation mystery by identifying an intra-system correlation in giant planet systems using Bayesian hierarchical modeling.
	
	\item Developed Gaussian process regression methodology for robust time-series correlated-noise removal, improving planetary signal recovery by 9X and adopted by 12 independent research groups.
	
	% Identified an intra-system correlation in giant planet systems using Bayesian hierarchical modeling. This work resolved a 30 year old mystery on the origin of Hot Jupiters.
	
	\item Lead and coordinated team of 16 scientists to perform quality control tasks; deliver science products; and produce peer-reviewed publications.

	
	\end{list2}		


\textbf{University of California, Los Angeles} \newline
\textit{Graduate Researcher} \hfill \textbf{September, 2016 - June, 2021}\newline
	\begin{list2}
		\vspace*{-5mm}
	\item Developed Python algorithm able to search highly contaminated time-series data and identify planet signals with 94\% reliability, resulting in the discovery of 372 planets: see \href{https://www.forbes.com/sites/jamiecartereurope/2021/06/11/we-found-372-new-alien-planets-using-a-long-dead-telescope-say-scientists/}{Forbes} \& \href{https://www.newsweek.com/astronomers-discover-366-new-worlds-gas-giants-kepler-k2-exoplanets-nasa-1653254}{Newsweek} press.
	
	\item Implemented random forest regression to characterize $\sim200,000$ stars, from training set of $\sim25,000$ well characterized stars, resulting in a 48\% accuracy improvement of existing methods.
	
	\item Validated 60 planet signal by cross-correlating their position with a Milky-way density map, resulting in a 0.1\% false-positive likelihood.
	
	\item Executed memory intensive simulation 12X faster than previous researchers by applying parallel code structure to open-source code.
	
	\item Derived Poisson point-process expansion, addressing sample bias issues attributed to the planet detection sequence, via order statistics, improving the accuracy of system architecture models by 36\%.
	
	\item Produced novel likelihood function for forward model comparison, using the Anderson-Darling EDF and a modified Poisson PDF, enabling the algorithm to converge 10X faster than existing methods.    
	
	% \item Awarded \$300k NASA grant for independent research.

    \end{list2}
\vspace*{-2mm}

\section{Software Development}
	\begin{list2}
	\item Produced open-source Python package able to parse exoplanet signals from statistical fluctuations in time-series data, used by $27$ independent research groups: see \href{https://github.com/jonzink/EDI_Vetter_unplugged}{EDIunplugged} on Github.
	
	\item Developed forward modeling software in Python and R (\href{https://github.com/jonzink/ExoMult}{ExoMult}), addresses intra-system correlations and sample biases, reducing the uncertainty of previous Earth-analog occurrence calculations by 63\%.
		
	\item Contributed to open source Python software through bug fixes and feature additions: see \href{https://github.com/California-Planet-Search/KPF-CPS}{KPF pipeline} on GitHub.
	\end{list2}
	

\vspace*{-2mm}
% \textbf{Dept. of Physics and Astronomy, Texas A\&M University} \hfill College Station, Texas\newline
% \textit{Ph.D Candidate} \hfill \textbf{August, 2010 - 2016}\newline
%     \begin{list2}
%     	\vspace*{-5mm}
%       \item Demonstrated that measurements from a planned large observation campaign could be improved by up to a factor of 3 over traditional statistical methods through the use of machine learning.
%       \item Implemented these machine learning methods and produced reliable results in a pilot survey of the real sky and under real-world conditions.
%     	% \item Collaborated with group members both in person, and through collaborative tools (e.g., GitHub, SVN).
%     	% \item Presented scientific results in high-impact, astrophysical journals and at international conferences.
%     \end{list2}
% \vspace*{-2mm}

% \textbf{The University of Tennessee}, Knoxville, Tennessee USA\newline
% \textit{Master's Candidate} \hfill \textbf{August, 2007 - 2009}\newline
%     \begin{list2}
%     	\vspace*{-5mm}
%       \item Implemented a C-based pipeline to process hundreds of GBs of simulation results. Including a computer vision algorithm to automatically analyze and compare results to expected targets.
%       \item Optimized simulation parameters using a genetic algorithm based search utilizing HPC (100k+ core) systems at the National Center for Computational Science, part of Oak Ridge National Laboratory
%     \end{list2}
% \vspace*{-3mm}

% \section{Awesome Projects}
% \textbf{Using Imaging to Infer Galaxy Properties}\newline
%     \begin{list2}
%     	\vspace*{-5mm}
%       \item Predicted galaxy chemical composition with $\sim$5\% error from pseudo-three color imaging, a result better than other current, similar efforts in the literature. Leveraged CNNs to analyze $\sim$150,000 images of galaxies.
%       \item Project start to publication: 4 months (typically $\sim$1.5 years). See: \href{https://github.com/boada/galaxy-cnns}{github.com/boada/galaxy-cnns}.
%     \end{list2}
%     \vspace*{-3mm}

% \textbf{Predicting Tournament Performance in Warmachine}\newline
%     \begin{list2}
%     	\vspace*{-5mm}
%     	\item Created an \href{https://en.wikipedia.org/wiki/Elo_rating_system}{Elo} based model to forecast the results of upcoming tournaments and identify potential upsets.
%     	\item Integrated predictions into a local community ranking system and forecasted $\sim$1800 tournament game results of the popular tabletop game using Python (e.g., Pandas).
%     \end{list2}
% \vspace*{-1mm}

\section{Skills}
\textbf{Machine Learning:} Random Forest, SVM, PCA, Clustering, Deep Learning, and Survival Analysis\\ 
\textbf{Statistical Methods:} Bayesian Analysis, Time-Series Analysis, Hypothesis Testing, Error Analysis, Monte Carlo Methods, Forward Modeling, EDF Testing, and Order Statistics\\
\textbf{Software and Computing:} Open Source Contributor, Python, R, C++, Java, SQL, HTML, PyMC, TensorFlow, Scikit-learn, and Object-Oriented Coding \\
% \textbf{Leadership:} Experience organizing and leading workshops and collaboration meetings, Teaching and mentoring junior team members, Eagle Scout. \\
\vspace*{-7mm}
% \noindent\rule{8cm}{0.4pt}

% \fullline{		% hrules only listen to \hoffset
%   \nointerlineskip	% so I have this code
%   \moveleft\hoffset\vbox{\hrule width\textwidth}
%   \nointerlineskip
% }


%section for two column education
\section{Education}

\begin{tabular}{@{}p{3in}p{3in}}
  \textbf{University of California}, Los Angeles
  \begin{list2}
  	\item Ph.D, Astrophysics, 2021
	\item B.S., Astrophysics, 2014
  \end{list2} &
\end{tabular}
\vspace*{-4mm}
\end{spacing}


\end{resume}
\end{document}
